\documentclass[article,type=bsc,colorback,accentcolor=tud9c]{tudthesis}
%\usepackage{ngerman}
\usepackage{hyperref}

\newcommand{\getmydate}{%
  \ifcase\month%
    \or Januar\or Februar\or M\"arz%
    \or April\or Mai\or Juni\or Juli%
    \or August\or September\or Oktober%
    \or November\or Dezember%
  \fi\ \number\year%
}

\begin{document}
  \thesistitle{Analysis of Methods for Background Execution in Modern Web Applications}%
    {Analyse von Verfahren für Hintergrundausführung in modernen Webanwendungen}
  \author{Yannick Reifschneider}
  \referee{}{}
  \department{Fachbereich Informatik}
  \group{Security Engineering}
  \dateofexam{\today}{\today}
  \tuprints{12345}{1234}
  \makethesistitle
  \affidavit{Yannick Reifschneider}

  \tableofcontents

  \newpage
  \section{Introduction}

  \subsection{Motivation}
  
  \subsection{The JavaScript execution model and the event loop}

  \newpage
  \section{Analysis of different background execution methods}

  \subsection{Timers}

  \subsection{Web workers}

  \subsection{Service workers}

  \subsection{Sensor readings}

  \newpage
  \section{Web browser behaviour regarding the different methods}

  \subsection{Desktop web browsers}

  \subsubsection{Google Chrome}

  \subsubsection{Mozilla Firefox}

  \subsubsection{Apple Safari}

  \subsection{Mobile web browsers}

  \subsubsection{iOS Mobile Safari}

  \subsubsection{Chrome for Android}

  \subsubsection{Firefox for Android}

  \newpage
  \section{Tracing of background execution on popular websites}

  Using the Alexa Top 100 Website list.
  
  \subsection{Method for measuring background execution}

  Maybe with puppeteer or with OpenWPM\footnote{\url{https://github.com/mozilla/OpenWPM}} or with simple hooking the JavaScript functions
  
  \subsection{Evaluation of findings}

  
  \newpage
  \section{Conclusion}

  

  

   

\end{document}
